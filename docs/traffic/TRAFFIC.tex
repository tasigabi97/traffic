\documentclass[12pt]{report}
%
\def\magyarOptions{defaults=hu-min}
%
\usepackage[magyar]{babel}
\usepackage{datatool}
\usepackage{t1enc}
\usepackage{times}
\usepackage{amsmath}
\usepackage{amssymb}
\usepackage{amsthm}
\usepackage{fancyhdr}
\usepackage{graphicx}
\usepackage{psfrag}
\usepackage{multirow}
\usepackage{xcolor}
\usepackage[colorlinks=true,linkcolor=blue,anchorcolor=red,citecolor=cyan,filecolor=red,menucolor=red,runcolor=red,urlcolor=cyan]{hyperref}
\usepackage{pifont}
\usepackage{nameref}
\usepackage{todonotes}
\usepackage{listings}
\usepackage{verbatim} 
\usepackage{graphicx}
\usepackage{wrapfig} 
\usepackage{subfigure}
\usepackage{etoolbox}
\usepackage{tabularx}
\usepackage{array}
\usepackage{xcolor}
\usepackage{listings}
\lstset{basicstyle=\ttfamily,
	showstringspaces=false,
	commentstyle=\color{red},
	keywordstyle=\color{blue}
}
%

%
\hoffset -1in
\voffset -1in
\oddsidemargin 20mm
\textwidth 180mm
\topmargin 1mm
\headheight 10mm
\headsep 5mm
\textheight 250mm
%
\graphicspath{ {./IMGS/} }
\lstset{language=Python}  
\renewcommand{\baselinestretch}{0.9}


\makeatletter
\patchcmd{\chapter}{\if@openright\cleardoublepage\else\clearpage\fi}{}{}{}
\makeatother


\newcommand{\img}[1]{\includegraphics[height=0.25\textheight]{#1}}
\begin{document}

\chapter{Linkek}
\begin{itemize}
\item NVIDIA docker install
\\\url{https://docs.nvidia.com/datacenter/cloud-native/container-toolkit/install-guide.html#installing-on-ubuntu-and-debian}

\item Docker run settings
\\\url{https://docs.nvidia.com/datacenter/cloud-native/container-toolkit/user-guide.html#nvidia-require-cuda}
\\TF/CUDA/cuDNN versions
\\\url{https://www.tensorflow.org/install/source#gpu}

\item Tensorflow containers
\\\url{https://hub.docker.com/r/tensorflow/tensorflow/tags?page=1&name=1.12}



\item Tensorflow gpu 
\\ Jó telepítés útmutató ubuntu fossa-ra:
\\ Nem szabad simán runnal elindítani mert akkor nem kapja meg a környezeti változókat a tensorflow. (venv) terminált kell használni helyette.
\\\url{https://towardsdatascience.com/installing-tensorflow-gpu-in-ubuntu-20-04-4ee3ca4cb75d}
\\ Másoknak sem jó 2.3-ra (nightly-ra igen) a 960M 
\\\url{https://github.com/tensorflow/tensorflow/issues/41990#issuecomment-683427929}

\item Stanford slides \\\url{http://cs231n.stanford.edu/slides/2017/cs231n_2017_lecture11.pdf#page=56}
\item Roi pooling example \\\url{https://medium.com/xplore-ai/implementing-attention-in-tensorflow-keras-using-roi-pooling-992508b6592b}
\item Fapados indiai előadás \\\url{https://www.youtube.com/watch?v=y6UmV8QwO9Q&list=PLkRkKTC6HZMy8smJGhhZ4HBIQgShLaTo8&ab_channel=ArdianUmam}
\item \url{https://lilianweng.github.io/lil-log/2017/12/31/object-recognition-for-dummies-part-3.html}
\item \url{https://arxiv.org/pdf/1504.08083.pdf}
\item \url{https://medium.com/@selfouly/part-2-fast-r-cnn-object-detection-7303e1988464}
\item \url{https://towardsdatascience.com/r-cnn-fast-r-cnn-faster-r-cnn-yolo-object-detection-algorithms-36d53571365e}
\item \url{https://papers.nips.cc/paper/5638-faster-r-cnn-towards-real-time-object-detection-with-region-proposal-networks.pdf}
\item \url{https://tryolabs.com/blog/2018/01/18/faster-r-cnn-down-the-rabbit-hole-of-modern-object-detection/}
\item \url{http://www.robots.ox.ac.uk/~tvg/publications/talks/fast-rcnn-slides.pdf}
\end{itemize}

\chapter{Argoverse}
\begin{itemize}
	\item \url{https://argoai.github.io/argoverse-api/}
	\item \url{https://github.com/argoai/argoverse-api}
\end{itemize}


\chapter{Kamera}
\url{https://www.dev47apps.com/droidcam/linux/}

\chapter{Átnézett adatbázis weboldalak}
\section{Adatbázis ajánló}
\url{http://homepages.inf.ed.ac.uk/rbf/CVonline/Imagedbase.htm#autodriving}
\begin{itemize}
	\item \url{http://www.cvl.isy.liu.se/en/research/datasets/amuse/}
	\\	\img{1}\img{2}
	\\\path{20130530_CVL_1_StraightForward.zip}
	\\Fizetős a nagy adatbázis
	\\De példák se jók
	%
	\item \url{http://apolloscape.auto/car_instance.html}
	\\	\img{3}\img{4}
	\\\img{5}
	\\\path{3d_car_instance_sample.tar.gz}
	\\5000 kép autó körvonallal, vannak hozzá programok
	%
	\item \url{http://apolloscape.auto/scene.html}
	\\\img{6}
	\\\path{road01_ins.tar.gz}
	\\150 000 kép (videó) pixelszintű szegmentálással
	\\25 kategória
	\\vannak hozzá programok
	%
	\item \url{http://apolloscape.auto/lane_segmentation.html}
	\\\img{7}
	\\\path{lane_marking_examples.tar.gz}
	\\110 000 kép (videó) pixelszintű útsáv szegmentálással
	\\van hozzá kiértékelő program
	%
	\item \url{https://www.argoverse.org/data.html}
	\\\img{8}\img{9}
	\\\img{10}
	\\\path{tracking_sample_v1.1.tar.gz}
	\\113 15-30 másodperces videó (képekben) befoglaló téglatestekkel és 15 kategóriával
	\\egyszerre 7 kamerával készült
	%
	\item \url{http://adas.cvc.uab.es/elektra/enigma-portfolio/cvc10-semantic-segmentation-dataset/}
	\\\img{11}
	\\\path{KITTI_SEMANTIC.zip}
	\\pixel szintű szegmentáció
	\\de csak 150 kép
	%
	\item \url{http://adas.cvc.uab.es/elektra/enigma-portfolio/item3/}
	\\\img{12}
	\\\path{CVC07.zip}
	\\2500 játékban sétáló gyalogos befoglaló négyzettel és pixels zintű szegmentációval
	\\nem valósághű 
	%
	\item \url{https://hci.iwr.uni-heidelberg.de/content/bosch-small-traffic-lights-dataset}
	\\\img{13}
	\\nem töltöttem le még
	\\13427 kép
	\\távoli jelzőlámpák befoglalónégyzettel
	%
	\item \url{https://drivingstereo-dataset.github.io/}
	\\\img{14}
	\\\path{DrivingStereo_demo_images.zip}
	\\csak sztereoképek nincs annotálás
	%
	\item \url{https://boxy-dataset.com/boxy/}
	\\\img{15}
	\\még nem töltöttem le
	\\rengeteg adat autókról
	\\koordináta tengelyekkel párhuzamos élű befoglaló téglatesttel
	\\\url{https://www.youtube.com/watch?v=HYO5Pthe3TE&feature=youtu.be}
	%
	\item \url{https://github.com/VRU-intention/casr}
	\\\img{16}
	\\kerékpárosok karjelzéseinek felismerése videóról
	%
	\item \url{http://rpg.ifi.uzh.ch/event_driving_datasets.html}
	\\\img{17}
	\\vezetés közben készített videók
	\\nincs annotáció
	%
	\item \url{http://robots.engin.umich.edu/SoftwareData/Ford}
	\\lézer kamera adatbázis
	\\nem hiszem hogy hasznos számunkra
	%
	\item \url{https://usa.honda-ri.com/H3D}
	\\lézer kamera adatbázis
	\\nem hiszem hogy hasznos számunkra
	%
	\item \url{https://github.com/facebookresearch/House3D}
	\\virtuális ház szegmentálva
	\\nem hiszem hogy hasznos számunkra
	%
	\item \url{http://idd.insaan.iiit.ac.in/dataset/details/}
	\\\img{18}
	\\\img{19}
	\\10 illetve 46 ezer kép
	\\pixel szintű szegmentáció illetve befoglaló téglalap
	%

	
	
\end{itemize}


\end{document}

